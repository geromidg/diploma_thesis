\label{Chapter5}

\chapter{Conclusion}

\section{Thesis Summary}

This thesis has considered the problem of perception for rovers that
aim to navigate autonomously in a planetary terrain.
We constructed an approach that is able to tackle the SLAM problem
in a relative context by making use of elevation mapping techniques using
3D point clouds and by employing a particle filter which utilizes a
scan matching method.
In addition, we extended the approach with the aim to correct errors
in position and orientation that accumulate over time using a template
matching method that locates the local elevation map in a globally referenced
one.

The proposed system was implemented as a standalone library and was
integrated with the software components of ESA's HDPR rover.
The integration with the HDPR's system was achieved using the ROCK
robotics framework.
This allowed the utilization of a rover with a sensor setup that bears close
resemblance to the one of the ExoMars mission.

Moreover, we demonstrated the developed system using collected data
from a Mars analogue (Chapter \ref{Chapter4}) and proved its
advantages and weaknesses in short and long range scenarios.
Specifically, the proposed method appeared to perform efficiently in
terrains that contained an average amount of elevation features (e.g. rocks).
However, in terrains that are characterized by the absence of such features,
the global pose correction mechanism fails to provide accurate matches.
Finally, we examined the influence the resolutions of the local and global
maps have on the matching technique.

\section{Contributions}

The main contribution of this thesis in terms of research,
is the development of a novel technique for minimizing the localization
drift with global map matching in addition to solving the SLAM problem
for a planetary rover.

For this purpose, we developed an open-source and research-oriented
C++ library with the following characteristics:

\begin{itemize}
    \item suitable for rough terrain navigation by utilizing
        2.5D (elevation) maps
    \item ability to correct the robot's global pose using map matching
        of local (robot-centric) and global (orbiter) elevation maps
    \item sensor-agnostic data registration using point clouds
        (enables the support for sensor such as lidar,
        stereo camera, ToF camera etc.)
    \item automatic sensor fusion when multiple point cloud inputs
        are provided without prior configuration
    \item adaptive design to fit the needs of different robots
        and applications.
\end{itemize}

\section{Directions for Future Extensions}

As we already mentioned, SLAM approaches on planetary rovers have an important
constraint which is the limited on board processing power.
Hence, it is important to design a lightweight system that requires as low
computations as possible.

With that in mind, a possible extension of the proposed particle filter
algorithm would be to model the space of the problem, i.e. localization,
in a discretized way instead of a continuous one.
This can be achieved by sampling particles in a 2D grid that has the same
resolution as the local map.
As a result, particles that fall in the same cell of the grid are eliminated
and are replaced by new particles.

Another direction for future work would be to exploit the uncertainty
of the robot's pose to improve the quality of the local map.
This can be accomplished using a neighborhood fusion method
\parencite{Fankhauser2014} by fusing nearby cells for each cell of the
map according to the variance of the robot's translation.
The variance values could be extracted directly from the particle filter,
by calculating the distribution of the particles, or by using the
motion model of the robot.

Finally, the global elevation map could be further utilized by using it
to initialize the cells of the local map.
This could provide a starting point for the fusion of new point cloud
data in the local map and increase the overall quality of the map.

\section{Applications}

Besides planetary scenarios, the proposed approach of this thesis
can also be used in Earth applications, with the obvious one being
autonomous driving vehicles.
Similarly to a planetary rover, self-driving cars require high
precision in terms of perception.

Although the usage of GPS data in such vehicles helps in the global
localization problem, its availability cannot be guaranteed at every
possible location, especially in non-urban areas.
A way to utilize the technique we developed in such applications,
would be to exploit global maps that have been a priori reconstructed
to contain information about the terrain's elevation.
This could provide self-driving vehicles with a mechanism to correct
possible drifts in position and orientation.

