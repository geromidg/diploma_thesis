\label{Chapter4}

\chapter{Experimental Validation}

\section{Scope of Experiments}

The purpose of the experiments is to validate the proposed system as
well as its robustness in various terrains and configurations.
Since the autonomous navigation of the robot is out of the scope of this
thesis, the mapping procedure is only indirectly validated as part of the
SLAM system.
Thus, the main focus of the experiments is the relative and absolute
localization techniques that were developed.

\subsection{Data Collection}

The performed experiments were executed by means of
\textit{Software In the Loop} (SIL) tests using precollected data.
The data were collected by the Planetary Robotics Lab team of the European
Space Agency as part of a two week test campaign that took
place in the Minas de San Jose desert in Tenerife, Spain with the aim to
validate the integration of the HDPR rover (Section \ref{hdpr_rover}) and
provide data for future research projects.

For that reason, the collected dataset contains samples from:
\begin{enumerate*}[label=(\roman*)]
        \item three stereo camera sensors
        \item IMU sensor
        \item point clouds reconstructed from drone imagery
        \item GPS sensor.
\end{enumerate*}
These samples are used both as input to our system and as ground truth
information.

The environment in which the test data were collected, was chosen with
the aim to bear close resemblance to rough planetary terrains,
specifically to the one of Mars.
In particular, various geological properties such as the distribution
and the shape of the rocks is similar to the Martian terrain.

% TODO: add figures showing the said environment/traverses from maps/drone

\subsection{Metrics}

To quantify the experimental results and evaluate the outputs of the system,
it is necessary to introduce certain metrics:
\begin{itemize}
    \item \textbf{Mean Square Error (MSE) of Pose Graph}:
        This is a straightforward metric for benchmarking SLAM algorithms
        and can be defined as
        \begin{equation}
            \varepsilon_{MSE} (\hat{x}_{1:T}) = \frac{1}{T}
            \sum\limits_{t=1}^T (\hat{x}_t \ominus x_t)^2
        \end{equation}
        where
        $\hat{x}_{1:T}$ is the estimated pose graph,
        $x_{1:T}$ is the ground truth pose graph,
        $T$ is the number of samples and
        $x_i \ominus x_j$ is the distance between two poses.

        % TODO(ref): "On Measuring the Accuracy of SLAM Algorithms" paper

    \item \textbf{Root Mean Square Deviation (RMSD) of Pose Graph}:
        Similarly to the MSE metric, this metric expresses the sample
        standard deviation of the differences between the estimated and the
        ground truth poses and can be defined as
        \begin{equation}
            \varepsilon_{RMSD} (\hat{x}_{1:T}) = \sqrt{\frac{1}{T}
            \sum\limits_{t=1}^T (y_t - \varepsilon_{MSE})^2}
        \end{equation}
        where $y_t$ is defined as $\hat{x}_t \ominus x_t$.

    \item (execution times)
\end{itemize}

\section{Experiments on Pose Estimation}

\subsection{Relative Localization Results}

Explain what the experiment is (test the accuracy of the particle filter)\\
Explain what is the expected accuracy of the estimation (1 cell size of local map)

\bigskip
\noindent
Add bird's eye view of XY plot comparing:
\begin{itemize}
    \item ground truth 2D position
    \item (Odometry)
    \item PF estimate
\end{itemize}

\noindent
Add plot showing the error vs the number of particles used\\
Add plot showing the mean variance of the estimate vs the number of particles used\\
Add table showing the execution time vs the numner of particles used\\
Add plot showing the error vs the resampling frequency

\bigskip
\noindent
Discuss the above results and what are the advantages/drawbacks while tuning each parameter

\section{Experiments on Global Map Matching}

\subsection{Absolute Localization Results}

Explain what the experiment is (test the drift correction on long range traverses)\\
Explain what is the expected accuracy of the correction (1 cell size of global map)

\bigskip
\noindent
Add bird's eye view of XY plot comparing:
\begin{itemize}
    \item ground truth 2D position
    \item (Odometry)
    \item PF estimate without global correction
    \item PF estimate with global correction
\end{itemize}

\bigskip
\noindent
Repeat test in environment with different distribution of features\\
(Repeat test in environment without features to show the limitation of the approach)

\bigskip
\noindent
Add table with the matching accuracy of the different experiments

\bigskip
\noindent
Discuss the threshold value that should be chosen on such environments

\subsection{Map Resolution Viability}

Explain what the experiment is (test under what resolutions can we expect a drift correction)

\bigskip
\noindent
Add figure showing local \& global maps with different resolution\\
Add table comparing global vs local map resolutions using the correction error (actual offset - matched location)

\bigskip
\noindent
Discuss the global and local map resolution of the current Mars missions and how these will change in the future (e.g. NASA is planning to have 0.25m orbiter map resolution by 2020)

