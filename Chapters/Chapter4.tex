\label{Chapter4}

\chapter{Experimental Validation}

\section{Scope of Experiments}

The purpose of the experiments is to validate the proposed system as
well as its robustness in various terrains and configurations.
Since the autonomous navigation of the robot is out of the scope of this
thesis, the mapping procedure is only indirectly validated as part of the
SLAM system.
Thus, the main focus of the experiments is the relative and absolute
localization techniques that were developed.

\subsection{Data Collection}

The performed experiments were executed by means of
\textit{Software In the Loop} (SIL) tests using precollected data.
The data were collected by the Planetary Robotics Lab team of the European
Space Agency as part of a two week test campaign that took
place in the Minas de San Jose desert in Tenerife, Spain with the aim to
validate the integration of the HDPR rover (Section \ref{hdpr_rover}) and
provide data for future research projects.

For that reason, the collected dataset contains samples from:
\begin{enumerate*}[label=(\roman*)]
        \item three stereo camera sensors
        \item IMU sensor
        \item point clouds reconstructed from drone imagery
        \item GPS sensor.
\end{enumerate*}
These samples are used both as input to our system and as ground truth
information.

The environment in which the test data were collected, was chosen with
the aim to bear close resemblance to rough planetary terrains,
specifically to the one of Mars.
In particular, various geological properties such as the distribution
and the shape of the rocks is similar to the Martian terrain.

% TODO: add figures showing the said environment/traverses from maps/drone

\subsection{Metrics} \label{metrics}

To quantify the experimental results and evaluate the outputs of the system,
it is necessary to introduce certain metrics:
\begin{itemize}
    \item \textbf{Mean square error (MSE) of pose graph}:
        This is a straightforward metric for benchmarking SLAM algorithms
        and can be defined as
        \begin{equation}
            \varepsilon_{MSE} (\hat{x}_{1:T}) = \frac{1}{T}
            \sum\limits_{t=1}^T (\hat{x}_t \ominus x_t)^2
        \end{equation}
        where
        $\hat{x}_{1:T}$ is the estimated pose graph,
        $x_{1:T}$ is the ground truth pose graph,
        $T$ is the number of samples and
        $x_i \ominus x_j$ is the distance between two poses.

        % TODO(ref): "On Measuring the Accuracy of SLAM Algorithms" paper

    \item \textbf{Root mean square deviation (RMSD) of pose graph}:
        Similarly to the MSE metric, this metric expresses the sample
        standard deviation of the differences between the estimated and the
        ground truth poses and can be defined as
        \begin{equation}
            \varepsilon_{RMSD} (\hat{x}_{1:T}) = \sqrt{\frac{1}{T}
            \sum\limits_{t=1}^T (y_t - \varepsilon_{MSE})^2}
        \end{equation}
        where $y_t$ is defined as $\hat{x}_t \ominus x_t$.

    % \item \textbf{Execution time of scan matching}:
    % \item \textbf{Execution time of map matching}:
\end{itemize}

\section{Experiments on Pose Estimation}

\subsection{Relative Localization Results}

The purpose of this experiment is to test the accuracy of the particle filter
in relative localization (short-range) scenarios.
To accomplish that, we will carry out each experiment in separate
environments and use different parameter sets with the aim of finding an
optimal configuration for the particle filter.

The main parameters that we will tune during the experiments are
\begin{enumerate*}[label=(\roman*)]
        \item the number of particles
        \item the resampling frequency
        \item the standard deviation of the Gaussian noise.
\end{enumerate*}
Regarding the environment, since the existence of features affects the
outcome of the particle filter's scan matching, it is necessary to
repeat the tests in terrains with distinct rock distributions.
Finally, the expected estimation error of the particle filter should
ideally be less than the size of one cell of the local map, so as to
ensure that the robot can distinguish the environment's obstacles with
a high-level of certainty.

\subsubsection{Scenario 1}

Figure \ref{fig:terrain_1} shows a view of the evaluated terrain of the first
scenario.
The terrain is depicted in the form of one dense point cloud which was
reconstructed from drone imagery.
% TODO: add 3D (not bird's eye view) of terrain from drone map PLY
The bird's eye view in Figure \ref{fig:bird_1} shows the traverse of the
robot in this scenario and compares the ground truth path, the one
from raw visual odometry and the one from the particle filter.
% TODO: add bird's eye view with: ground truth/Odometry/PF estimate
% TODO: quickly discuss result

In particular, the effect of the number of particles used has on
the outcome is shown in the comparison plots in Figures
\ref{fig:mean_error_vs_particles} and \ref{fig:std_error_vs_particles}.
% TODO: add plot showing the mean error vs particles
% TODO: add plot showing the std error vs particles
% TODO: quickly discuss result

Furthermore, the plots in Figures \ref{fig:mean_error_vs_resampling} and
\ref{fig:std_error_vs_resampling} represent the same errors as the previous
plots, with the difference that the number of particles is fixed to the
value of 50 and the resampling frequency is the variable parameter.
% TODO: add plot showing the mean error vs resampling
% TODO: add plot showing the std error vs resampling
% TODO: quickly discuss result

For the purpose of further quantifying the results of this scenario altogether,
we calculate the $MSE$ and $RMSD$ metrics (defined in Section \ref{metrics})
for the parameter sets that were used to generate the above plots.
The values of these metrics are presented in Table \ref{table:metrics_1}.
% TODO: add table with MSE/RMSD vs {1particle+5resampling, 50p+5r, 100p+5r, ..}
% TODO: quickly discuss result

% TODO: discuss the above results and what are the advantages/drawbacks while tuning each parameter

\subsubsection{Scenario 2}

In the second scenario, a more sparse terrain was evaluated with a focus on
quantifying the efficiency of the developed scan matching technique, which
is directly affected by the existence of features.
Figure \ref{fig:terrain_2} shows the aforementioned terrain and
Figure \ref{fig:bird_2} shows the robot's traverse in it.

% TODO

\section{Experiments on Global Map Matching}

\subsection{Absolute Localization Results}

Explain what the experiment is (test the drift correction on long range traverses)\\
Explain what is the expected accuracy of the correction (1 cell size of global map)

%TODO: add bird's eye view of XY plot comparing:
%   ground truth 2D position
%   (Odometry)
%   PF estimate without global correction
%   PF estimate with global correction

% TODO: pepeat test in environment with different distribution of features
% TODO: (repeat test in environment without features to show the limitation of the approach)

% TODO: add table with the matching accuracy of the different experiments

% TODO: discuss the threshold value that should be chosen on such environments

\subsection{Map Resolution Viability}

Explain what the experiment is (test under what resolutions can we expect a drift correction)

% TODO: add figure showing local \& global maps with different resolution
% TODO: add table comparing global vs local map resolutions using the correction error (actual offset - matched location)

% TODO: discuss the global and local map resolution of the current Mars missions and how these will change in the future (e.g. NASA is planning to have 0.25m orbiter map resolution by 2020)

