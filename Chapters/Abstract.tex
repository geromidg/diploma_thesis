\begin{abstract}%

Rover autonomy is of paramount significance in a planetary exploration
mission, since low supervision from the ground station can directly
increase the scientific outcomes.
However, the autonomous long range navigation of robotic vehicles
raises several challenges, especially in such context where the
communication to Earth is limited.
An essential skill such vehicle must possess in order to navigate in
an unknown environment safely is the accurate perception of its surroundings.

This thesis proposes an approach for real-time Simultaneous Localization and
Mapping (SLAM) in extreme planetary terrains.
By employing scan matching, particle filter and elevation mapping techniques,
we construct a system that is able to provide an accurate pose estimation
and a high resolution representation of the environment which can be
utilized for navigation purposes.

Moreover, the current thesis addresses the problem of absolute localization
that emerges due to the accumulated drift in position and orientation.
Building upon existing image matching techniques, we formulate a novel
method that is capable of locating a local (robot-centric) map
in an a priori global map with the aim to correct the robot's pose in
a global reference frame.
Additionally, the developed matching technique is evaluated against possible
local and global map resolution values so as to provide a baseline
for future applications.

Finally, we implement and integrate our approach on a planetary rover
testbed as a means to validate the performance of the unified system.
The experimental validation is performed in long range scenarios in
a terrain that bears close resemblance to a planetary one.

\end{abstract}

