\label{Chapter1}

\chapter{Introduction}

\section{Motivation}

Mention the scope, its state, the current issues and what is needed to
solve them

\begin{itemize}
    \item Scope: planetary exploration (for scientific purposes)
    \item State: past and current missions, their purpose and outcome/state
    \item Issues: no constant communication, manual command sequence, unsafe/upredictable conditions
    \item Needed: real-time and low-supervision system/platform
\end{itemize}

\section{Problem Statement}

Mention the environment and the robot's equipment and purpose\\
Mention that the problem is given X inputs to calculate Y outputs

\begin{itemize}
    \item Environment: extreme planetary terrains with anomalies (rocks and craters) and high elevation variance
    \item Equipment: mechanical and sensor configuration
    \item Purpose: autonomous (low-supervision) navigation
    \item Inputs: odometry pose, sensor inputs, imu, global map
    \item Outputs: corrected pose, local elevation map
\end{itemize}

\subsection{Presumptions}

Explain what presumptions are made to the problem

\begin{itemize}
    \item Initial global position is known
    \item Environment is static (not true for autonomous driving applications of algorithm)
    \item Environment is unknown (no high resolution a priori maps) (not true for autonomous driving applications of algorithm)
    \item Terrain is single layered (no bridges etc.)
\end{itemize}

\section{Literature Review}

\subsection{SLAM}

\begin{itemize}
    \item  What is SLAM
        \begin{itemize}
            \item typical explanation in literature
            \item categories
        \end{itemize}
    \item How is SLAM implemented
        \begin{itemize}
            \item volumetric/feature based approaches
            \item grid-based fast slam
            \item etc. etc. etc.
        \end{itemize}
\end{itemize}

\subsection{Planetary Absolute Localization}

Mention how is absolute localization achieved in planetary applications

\begin{itemize}
    \item Feature based approaches
    \item Skyline based approaches
    \item map based approaches
\end{itemize}

\section{Thesis Objectives and Organization}

\subsection{Research Objectives}

Mention what are the main objectives in terms of research

\begin{itemize}
    \item Develop a novel technique for minimizing drift in global localization with global map matching and an execution strategy (criteria)
    \item Determine under what circumstances (i.e. resolution, local and global) can the orbital imagery be useful for SLAM
    \item Examine and quantify the gains (better localization, by providing a navigable map with the resolution and dense distribution restrictions that come with it) and loses (processing overhead)
\end{itemize}

\subsection{Organization}

Explain how this thesis is structured

