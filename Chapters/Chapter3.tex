\label{Chapter3}

\chapter{System Implementation}

\section{Library}

Mention GA SLAM library and tools used to create it etc.

\subsection{Concurrency}

Explain the algorithm's processing limitations and the circumstances under which it can run\\
Explain how concurrency can help and how it's implemented\\
Add timing diagram explaining the issues and how they are solved

\subsection{Robotic Software Framework}

Explain briefly ROCK, its tools, workflow and advantages/disadvantages\\
Explain briefly ROS etc.\\
Mention that the standalone library has interface to both frameworks

\subsection{Orbiter Data Preprocessing}

Explain that orbital imagery is emulated using drone imagery\\
Explain quickly how a 3D point cloud is reconstructed from the orbital imagery\\
Mention NASA's HiRISE (High Resolution Imaging Science Experiment)

\bigskip
\noindent
Explain the steps for creating a global map from orbital point cloud:
\begin{enumerate}
    \item  Voxelize point cloud to the desired resolution
    \item Crop the point cloud in the order of magnitude as local map's size
    \item Smooth the point cloud using a SOR filter
    \item Transform point cloud to robot's pose (using the initial absolute position)
\end{enumerate}

\noindent
Add figures of orbiter point cloud (raw \& processed) and the orbiter map\\
Add figures from NASA's HiRISE depicting actual Mars point clouds

\section{System Architecture}

Explain how the algorithm can be integrated in a system and what components are needed (stereo, odometry etc.)\\
Add component diagram showing the architecture and how the different components are integrated\\

\section{Planetary Rover Testbed}

Explain lab's main rovers (HDPR \& Exoter) and their mechanical/sensor configuration\\
Add figure depicting and explaining the rover(s)

